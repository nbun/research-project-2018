% !TEX TS-program = xelatex
\documentclass{beamer}
\usepackage{ulem}
\usepackage{proof}
\usepackage{bussproofs}
\usepackage{amsmath}
\usepackage{amssymb}
\usepackage{minted}
\usepackage{subcaption}
\usepackage{natbib}
\usepackage{graphicx}
\usepackage{scrextend}
\usepackage{appendixnumberbeamer}

\usemintedstyle[Haskell]{trac, fontsize=\small}
\usemintedstyle[Coq]{fontsize=\footnotesize}
\usemintedstyle[Text]{fontsize=\footnotesize}
\newcommand{\haskellinline}[1]{\mintinline{Haskell}{#1}}
\newcommand{\coqinline}[1]{\mintinline{Coq}{#1}}

\usetheme[progressbar=foot]{metropolis}

\usepackage[german]{babel}
\usepackage[utf8]{inputenc}

%\setbeamersize{text margin left=1.5em,text margin right=1.5em}

\title{Synthesizing Set Functions: Eine prototypische Implementierung}
\subject{Synthesizing Set Functions: Eine prototypische Implementierung}
\date{23. Januar 2019}
\author{Niels Bunkenburg}
\institute{ 
	Arbeitsgruppe für Programmiersprachen und Übersetzerkonstruktion \par
	Institut für Informatik \par
	Christian-Albrechts-Universität zu Kiel}

\begin{document}

\begin{frame}
  \titlepage
\end{frame}

\begin{frame}[fragile]{Aufgabe}
\begin{itemize}
\item \alert{Set Functions} im KiCS2 verhalten sich anders als im PAKCS
\vspace{.5em}
\item Set Functions kapseln Nichtdeterminismus einer Funktion
\begin{minted}[escapeinside=<>]{Haskell}
coin = 0 ? 1 <$\Rightarrow$> coin<$_S$> = {0,1}
\end{minted}
\vspace{.5em}
\item Nichtdeterminismus in Argumenten wird nicht gekapselt
\begin{minted}[escapeinside=<>]{Haskell}
double x = x + x <$\Rightarrow$> double<$_S$> coin = {0}, {2}
\end{minted}
\vspace{.5em}
\item Neuer Ansatz: Synthesizing Set Functions \citep{synsf}
\end{itemize}
\end{frame}

\begin{frame}[fragile]{Idee: Plural Functions}
\begin{itemize}
\item Set Functions bestehen aus \alert{Plural Functions}
\item Plural Function für $f :: a \rightarrow b$ ist  $f_P :: \{a\} \rightarrow 
\{b\}$
\item Implementierung von $\{.\}$ als Suchbaum
\begin{minted}{Haskell}
data ST a = Val a | Fail | Choice (ST a) (ST a)
\end{minted}
\item Anpassung der Ausdrücke
\begin{minted}{Haskell}
applyST :: (a -> ST b) -> ST a -> ST b

notP :: ST Bool -> ST Bool
notP = applyST $ \x ->
                 case x of False -> Val True
                           True  -> Val False
\end{minted}
\end{itemize}
\end{frame}

\begin{frame}[fragile]{Datentypen}
\begin{itemize}
\item Komponenten in Datentypen werden zu Suchbäumen
\begin{minted}{Haskell}
data STList a = Nil | Cons (ST a) (ST (STList a))
\end{minted}
\item Normalform ("Hochziehen" von Nichtdeterminismus)
\begin{minted}{Haskell}
class NF a where
  nf :: a -> ST a
\end{minted}
\item Konvertierung zwischen ursprünglichem und ST-Datentyp
\begin{minted}{Haskell}
class ConvertST a b where
  toValST :: a -> b
  fromValST :: b -> a
\end{minted}
\end{itemize}
\end{frame}

\begin{frame}{Implementierung}
Transformationsphasen
\begin{enumerate}
\item Aufsammeln der verwendeten Definitionen
\item Lifting von verschachtelten Case-Ausdrücken
\item Plural Function Transformation (Funktionstyp und Regeln) $\Rightarrow$ 
Vorkommende Datentypen werden transformiert
\item Normalform Instanzen
\item ConvertST Instanzen
\item Zusammenbauen der Set Function
\end{enumerate}
\end{frame}

\begin{frame}{Ergebnisse}
\begin{itemize}
\item Implementierung der Transformation grundsätzlich möglich
\item Einschränkungen
\begin{itemize}
\item Verschachtelte Set Functions
\item Polymorphie/Higher-Order/Typsynonyme
\item Externe Funktionen
\item Freie Variablen
\end{itemize}
\item Effizienz und Korrektheit?
\item Integration in Präprozessor?
\end{itemize}
\end{frame}

\begin{frame}
\bibliographystyle{plainnat}
\bibliography{ResearchProject18}
\end{frame}
\end{document}


